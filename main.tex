\documentclass{article}
\usepackage[left=1in,right=1in]{geometry}
\usepackage{subfiles}
\usepackage{amsmath, amssymb, stmaryrd, verbatim} % math symbols
\usepackage{amsthm} % thm environment
\usepackage{mdframed} % Customizable Boxes
\usepackage{hyperref,nameref,cleveref,enumitem} % for references, hyperlinks
\usepackage[dvipsnames]{xcolor} % Fancy Colours
\usepackage{mathrsfs} % Fancy font
\usepackage{bbm} % mathbb numerals
\usepackage{tikz, tikz-cd, float} % Commutative Diagrams
\usetikzlibrary{decorations.pathmorphing} % for squiggly arrows in tikzcd
\usepackage{perpage}
\usepackage{parskip} % So that paragraphs look nice
\usepackage{ifthen,xargs} % For defining better commands
\usepackage[T1]{fontenc}
\usepackage[utf8]{inputenc}
\usepackage{tgpagella}
\usepackage{cancel}

% for blackboard Delta
\usepackage{bbm}
\usepackage[bbgreekl]{mathbbol}

% Bibliography
\usepackage{url}
\usepackage{biblatex}

\addbibresource{mybib.bib}

% Shortcuts

% % Local to this project
\renewcommand\labelitemi{--} % Makes itemize use dashes instead of bullets
\newcommand{\AN}{\mathrm{An}}
\newcommand{\MONO}{\mathrm{Mono}}
\newcommand{\mono}{\hookrightarrow}
\newcommand{\epi}{\twoheadrightarrow}
\newcommand{\cof}{\rightarrowtail}
\newcommand{\bb}[1]{\mathbb{#1}}
\newcommand{\ISO}{\mathrm{Iso}\,}
\newcommand{\SSET}{\mathbf{sSet}}
\newcommand{\MOR}{\mathbf{Mor}}
\newcommand{\UNIT}{\mathrm{Unit}}
\newcommand{\SUB}{\mathbf{Sub}}
\newcommand{\AFF}{\mathbf{Aff}}
\newcommand{\QCOH}{\mathbf{QCoh}}
\newcommand{\SPEC}{\mathrm{Spec}}
\newcommand{\LIFT}{\boxslash}
\newcommand{\LLIFT}[1]{{}^\boxslash #1}
\newcommand{\RLIFT}[1]{{#1}^\boxslash}
\newcommand{\PTYPE}{\mathrm{PreType}}
\newcommand{\HO}{\mathrm{Ho}}
\newcommand{\REFL}{\texttt{refl}}
\DeclareMathSymbol\bbDe  \mathord{bbold}{"01} % blackboard Delta
\newcommand{\RH}{\mathrm{RH}}
\newcommand{\LH}{\mathrm{LH}}
\newcommand{\IH}{\mathrm{IH}}

\newcommand{\SK}{\mathrm{Sk}}
\newcommand{\SP}{\mathrm{Sp}}
\newcommand{\CELL}{\mathrm{Cell}}
\newcommand{\COSDR}{\mathrm{coSDR}}
\newcommand{\KAN}{\mathrm{Kan}}
\newcommand{\EX}{\mathrm{Ex}}
\newcommand{\RETRACT}{\mathrm{Retract}}
\newcommand{\QCAT}{\mathbf{qCat}}

% % Misc
\newcommand{\brkt}[1]{\left(#1\right)}
\newcommand{\sqbrkt}[1]{\left[#1\right]}
\newcommand{\dash}{\text{-}}

% % Logic
\renewcommand{\implies}{\Rightarrow}
\renewcommand{\iff}{\Leftrightarrow}
\newcommand{\limplies}{\Leftarrow}
\newcommand{\NOT}{\neg\,}
\newcommand{\AND}{\, \land \,}
\newcommand{\OR}{\, \lor \,}
\newenvironment{forward}{($\implies$)}{}
\newenvironment{backward}{($\limplies$)}{}

% % Sets
\DeclareMathOperator{\supp}{supp}
\newcommand{\set}[1]{\left\{#1\right\}}
\newcommand{\st}{\,|\,}
\newcommand{\minus}{\setminus}
\newcommand{\subs}{\subseteq}
\newcommand{\ssubs}{\subsetneq}
\newcommand{\sups}{\supseteq}
\newcommand{\ssups}{\supset}
\DeclareMathOperator{\im}{Im}
\newcommand{\nothing}{\varnothing}
\DeclareMathOperator{\join}{\sqcup}
\DeclareMathOperator{\meet}{\sqcap}

% % Greek 
\newcommand{\al}{\alpha}
\newcommand{\be}{\beta}
\newcommand{\ga}{\gamma}
\newcommand{\de}{\delta}
\newcommand{\ep}{\varepsilon}
\newcommand{\ph}{\varphi}
\newcommand{\io}{\iota}
\newcommand{\ka}{\kappa}
\newcommand{\la}{\lambda}
\newcommand{\om}{\omega}
\newcommand{\si}{\sigma}

\newcommand{\Ga}{\Gamma}
\newcommand{\De}{\Delta}
\newcommand{\Th}{\Theta}
\newcommand{\La}{\Lambda}
\newcommand{\Si}{\Sigma}
\newcommand{\Om}{\Omega}

% % Mathbb
\newcommand{\A}{\mathbb{A}}
\newcommand{\C}{\mathbb{C}}
\newcommand{\F}{\mathbb{F}}
\newcommand{\G}{\mathbb{G}}
\newcommand{\I}{\mathbb{I}}
\newcommand{\J}{\mathbb{J}}
\newcommand{\M}{\mathbb{M}}
\newcommand{\N}{\mathbb{N}}
\renewcommand{\P}{\mathbb{P}}
\newcommand{\Q}{\mathbb{Q}}
\newcommand{\R}{\mathbb{R}}
\newcommand{\U}{\mathbb{U}}
\newcommand{\V}{\mathbb{V}}
\newcommand{\Z}{\mathbb{Z}}

% % Mathcal
\newcommand{\BB}{\mathcal{B}}
\newcommand{\CC}{\mathcal{C}}
\newcommand{\DD}{\mathcal{D}}
\newcommand{\EE}{\mathcal{E}}
\newcommand{\FF}{\mathcal{F}}
\newcommand{\GG}{\mathcal{G}}
\newcommand{\HH}{\mathcal{H}}
\newcommand{\II}{\mathcal{I}}
\newcommand{\JJ}{\mathcal{J}}
\newcommand{\KK}{\mathcal{K}}
\newcommand{\LL}{\mathcal{L}}
\newcommand{\MM}{\mathcal{M}}
\newcommand{\NN}{\mathcal{N}}
\newcommand{\OO}{\mathcal{O}}
\newcommand{\PP}{\mathcal{P}}
\newcommand{\QQ}{\mathcal{Q}}
\newcommand{\RR}{\mathcal{R}}
\newcommand{\TT}{\mathcal{T}}
\newcommand{\UU}{\mathcal{U}}
\newcommand{\VV}{\mathcal{V}}
\newcommand{\WW}{\mathcal{W}}
\newcommand{\XX}{\mathcal{X}}
\newcommand{\YY}{\mathcal{Y}}
\newcommand{\ZZ}{\mathcal{Z}}

% % Mathfrak
\newcommand{\f}[1]{\mathfrak{#1}}

% % Mathrsfs
\newcommand{\s}[1]{\mathscr{#1}}

% % Category Theory
\DeclareMathOperator{\OBJ}{Obj}
\DeclareMathOperator{\END}{End}
\DeclareMathOperator{\AUT}{Aut}
\newcommand{\CAT}{\mathbf{Cat}}
\newcommand{\SET}{\mathbf{Set}}
\newcommand{\TOP}{\mathbf{Top}}
\newcommand{\MON}{\mathbf{Mon}}
\newcommand{\GRP}{\mathbf{Grp}}
\newcommand{\AB}{\mathbf{Ab}}
\newcommand{\RING}{\mathbf{Ring}}
\newcommand{\CRING}{\mathbf{CRing}}
\newcommand{\MOD}{\mathbf{Mod}}
\newcommand{\VEC}{\mathbf{Vec}}
\newcommand{\ALG}{\mathbf{Alg}}
\newcommand{\PSH}{\mathbf{PSh}}
\newcommand{\SH}{\mathbf{Sh}}
\newcommand{\ORD}{\mathbf{Ord}}
\newcommand{\POSET}{\mathbf{PoSet}}
\newcommand{\id}[1]{\mathbbm{1}_{#1}}
\newcommand{\map}[2]{\yrightarrow[#2][2.5pt]{#1}[-1pt]}
\newcommand{\iso}{\cong}
\newcommand{\op}{^{op}}
\newcommand{\darrow}{\downarrow}
\newcommand{\LIM}{\varprojlim}
\newcommand{\COLIM}{\varinjlim}
\DeclareMathOperator{\coker}{coker}
\newcommand{\fall}[2]{\downarrow_{#2}^{#1}}
\newcommand{\lift}[2]{\uparrow_{#1}^{#2}}

% % Algebra
\newcommand{\nsub}{\trianglelefteq}
\newcommand{\inv}{^{-1}}
\newcommand{\dvd}{\,|\,}
\DeclareMathOperator{\ev}{ev}

% % Analysis
\newcommand{\abs}[1]{\left\vert #1 \right\vert}
\newcommand{\norm}[1]{\left\Vert #1 \right\Vert}
\renewcommand{\bar}[1]{\overline{#1}}
\newcommand{\<}{\langle}
\renewcommand{\>}{\rangle}
\renewcommand{\hat}[1]{\widehat{#1}}
\renewcommand{\check}[1]{\widecheck{#1}}
\newcommand{\dsum}[2]{\sum_{#1}^{#2}}
\newcommand{\dprod}[2]{\prod_{#1}^{#2}}
\newcommand{\del}[2]{\frac{\partial#1}{\partial#2}}
\newcommand{\res}[2]{{% we make the whole thing an ordinary symbol
  \left.\kern-\nulldelimiterspace % automatically resize the bar with \right
  #1 % the function
  %\vphantom{\big|} % pretend it's a little taller at normal size
  \right|_{#2} % this is the delimiter
  }}

% % Galois
\DeclareMathOperator{\gal}{Gal}
\DeclareMathOperator{\Orb}{Orb}
\DeclareMathOperator{\Stab}{Stab}
\newcommand{\emb}[3]{\mathrm{Emb}_{#1}(#2, #3)}
\newcommand{\Char}[1]{\mathrm{Char}#1}

%% code from mathabx.sty and mathabx.dcl to get some symbols from mathabx
\DeclareFontFamily{U}{mathx}{\hyphenchar\font45}
\DeclareFontShape{U}{mathx}{m}{n}{
      <5> <6> <7> <8> <9> <10>
      <10.95> <12> <14.4> <17.28> <20.74> <24.88>
      mathx10
      }{}
\DeclareSymbolFont{mathx}{U}{mathx}{m}{n}
\DeclareFontSubstitution{U}{mathx}{m}{n}
\DeclareMathAccent{\widecheck}{0}{mathx}{"71}

% Arrows with text above and below with adjustable displacement
% (Stolen from Stackexchange)
\newcommandx{\yaHelper}[2][1=\empty]{
\ifthenelse{\equal{#1}{\empty}}
  % no offset
  { \ensuremath{ \scriptstyle{ #2 } } } 
  % with offset
  { \raisebox{ #1 }[0pt][0pt]{ \ensuremath{ \scriptstyle{ #2 } } } }  
}

\newcommandx{\yrightarrow}[4][1=\empty, 2=\empty, 4=\empty, usedefault=@]{
  \ifthenelse{\equal{#2}{\empty}}
  % there's no text below
  { \xrightarrow{ \protect{ \yaHelper[ #4 ]{ #3 } } } } 
  % there's text below
  {
    \xrightarrow[ \protect{ \yaHelper[ #2 ]{ #1 } } ]
    { \protect{ \yaHelper[ #4 ]{ #3 } } } 
  } 
}

% xcolor
\definecolor{darkgrey}{gray}{0.10}
\definecolor{lightgrey}{gray}{0.30}
\definecolor{slightgrey}{gray}{0.80}
\definecolor{softblue}{RGB}{30,100,200}

% hyperref
\hypersetup{
      colorlinks = true,
      linkcolor = {softblue},
      citecolor = {blue}
}

\newcommand{\link}[1]{\hypertarget{#1}{}}
\newcommand{\linkto}[2]{\hyperlink{#1}{#2}}

% Perpage
\MakePerPage{footnote}

% Theorems

% % custom theoremstyles
\newtheoremstyle{definitionstyle}
{5pt}% above thm
{0pt}% below thm
{}% body font
{}% space to indent
{\bf}% head font
{\vspace{1mm}}% punctuation between head and body
{\newline}% space after head
{\thmname{#1}\thmnote{\,\,--\,\,#3}}

\newtheoremstyle{exercisestyle}%
{5pt}% above thm
{0pt}% below thm
{\it}% body font
{}% space to indent
{\it}% head font
{.}% punctuation between head and body
{ }% space after head
{\thmname{#1}\thmnote{ (#3)}}

\newtheoremstyle{examplestyle}%
{5pt}% above thm
{0pt}% below thm
{\it}% body font
{}% space to indent
{\it}% head font
{.}% punctuation between head and body
{\newline}% space after head
{\thmname{#1}\thmnote{ (#3)}}

\newtheoremstyle{remarkstyle}%
{5pt}% above thm
{0pt}% below thm
{}% body font
{}% space to indent
{\it}% head font
{.}% punctuation between head and body
{ }% space after head
{\thmname{#1}\thmnote{\,\,--\,\,#3}}

% Custom Environments

% % Theorem environments

\theoremstyle{definitionstyle}
\newmdtheoremenv[
    linewidth = 2pt,
    leftmargin = 0pt,
    rightmargin = 0pt,
    linecolor = darkgrey,
    topline = false,
    bottomline = false,
    rightline = false,
    footnoteinside = true
]{dfn}{Definition}
\newmdtheoremenv[
    linewidth = 2 pt,
    leftmargin = 0pt,
    rightmargin = 0pt,
    linecolor = darkgrey,
    topline = false,
    bottomline = false,
    rightline = false,
    footnoteinside = true
]{prop}{Proposition}
\newmdtheoremenv[
    linewidth = 2 pt,
    leftmargin = 0pt,
    rightmargin = 0pt,
    linecolor = darkgrey,
    topline = false,
    bottomline = false,
    rightline = false,
    footnoteinside = true
]{cor}{Corollary}

\theoremstyle{exercisestyle}
\newmdtheoremenv[
    linewidth = 0.7 pt,
    leftmargin = 20pt,
    rightmargin = 0pt,
    linecolor = darkgrey,
    topline = false,
    bottomline = false,
    rightline = false,
    footnoteinside = true
]{ex}{Exercise}
\newmdtheoremenv[
    linewidth = 0.7 pt,
    leftmargin = 20pt,
    rightmargin = 0pt,
    linecolor = darkgrey,
    topline = false,
    bottomline = false,
    rightline = false,
    footnoteinside = true
]{lem}{Lemma}

\theoremstyle{examplestyle}
\newmdtheoremenv[
    linewidth = 0.7 pt,
    leftmargin = 20pt,
    rightmargin = 0pt,
    linecolor = darkgrey,
    topline = false,
    bottomline = false,
    rightline = false,
    footnoteinside = true
]{eg}{Example}
\newmdtheoremenv[
    linewidth = 0.7 pt,
    leftmargin = 20pt,
    rightmargin = 0pt,
    linecolor = darkgrey,
    topline = false,
    bottomline = false,
    rightline = false,
    footnoteinside = true
]{ceg}{Counter Example}

\theoremstyle{remarkstyle}
\newtheorem{rmk}{Remark}
\newtheorem{intuit}{Intuition}
\newtheorem{hole}{Missing Understanding}

\newenvironment{proof1}{
  \begin{proof}\renewcommand\qedsymbol{$\blacksquare$}
}{
  \end{proof}
} % Proofs ending with black qedsymbol 

% % tikzcd diagram 
\newenvironment{cd}{
    \begin{figure}[H]
    \centering
    \begin{tikzcd}
}{
    \end{tikzcd}
    \end{figure}
}

% tikzcd
% % Substituting symbols for arrows in tikz comm-diagrams.
\tikzset{
  symbol/.style={
    draw=none,
    every to/.append style={
      edge node={node [sloped, allow upside down, auto=false]{$#1$}}}
  }
}

\begin{document}

\title{Notes on Presheaves as Discrete Fibrations}

\author{Ken Lee}
\date{2021 Summer}
\maketitle
\tableofcontents

% rmk. what an (∞,1)-category ``should'' be
% - infinity groupoid of a topological space
% - simplicial categories : stack of modules

\section{Presheaves - Objects probeable by a category of test objects}

\begin{rmk}[Presheaves as Probeable Objects]
  
  A general philosophy in mathematics is to
  study objects by ``probing them with test objects''
  which we understand,
  where by ``probing'' we mean mapping these ``test objects'' into them.
  The ``test objects'' usually form a (small) category,
  which for this discussion let's call $C$.
  Our main example throughout will be $C = $``the category of triangles''.
  A ``presheaf over $C$'' is exactly 
  ``an object that is probeable by test objects in $C$''.
  The steps to formalising this statement are : 
  \begin{enumerate}
    \item If $X$ is an object 
    ``that can be mapped into by test objects in $C$'',
    then we ought to have a set $X(c)$ for each $c$ in $C$,
    which we think of as the set of maps $c \to X$.
    We will call these \emph{naive morphisms} from $c$ to $X$.
    \item if we have a morphism $f : d \to c$ in $C$,
    then \emph{restricting along $f$} should give a way
    of turning naive morphisms from $c$ to $X$ into
    naive morphisms from $d$ to $X$.
    \begin{cd}
      d & c \\
      & X
      \arrow["f", from=1-1, to=1-2]
      \arrow["x", from=1-2, to=2-2]
      \arrow["xf"', from=1-1, to=2-2]
    \end{cd}
    In other words, for each morphism $f : d \to c$ in $C$,
    we want the data of a morphism of sets $X(c) \to X(d)$.
    Furthermore, we should have 
    \begin{cd}
      e & d & c & c & {} \\
      && X
      \arrow["f", from=1-2, to=1-3]
      \arrow["x", from=1-3, to=2-3]
      \arrow["g", from=1-1, to=1-2]
      \arrow["{xfg = (xf)g}"', from=1-1, to=2-3]
      \arrow["{\id{c}}"{swap},from=1-4, to=1-3]
      \arrow["{x = x \id{c}}", from=1-4, to=2-3]
    \end{cd}
    This gives exactly the definition of a presheaf over $C$
    as a functor $C\op \to \SET$.
    \item The intuition that the morphisms of presheaves \emph{should} be
    natural transformations of functors comes from the idea :
    ``if $f : X \to Y$ is a morphism between objects probeable 
    by test objects in $C$, then
    \emph{composing along $f$} should give a way of turning 
    naive morphisms $c \to X$ into naive morphisms $c \to Y$''.
    Under this point of view, naturality is tautological.
  \end{enumerate}

  Nice as the above interpretation is,
  this is \emph{not} the definition of presheaves we will take.
  The following proposition gives an alternate definition,
  which can be seen as a generalisation of 
  \emph{covering spaces} from algebraic topology.
  Indeed, the classical theory of covering spaces
  can be seen as establishing conditions on the base topological space
  for which discrete fibrations over the (dual of the) fundamental groupoid
  correspond to actual topological spaces over the base space.
\end{rmk}

\begin{prop}[``A bundle of sets is a bundle of sets'']

  Let $ C$ be a (small) category.

  For $(p : X \to  C)$ in $\CAT /  C$, 
  we say $p$ is a \emph{discrete fibration} when
  any of the following equivalent conditions are met :
  \begin{enumerate}
    \item any commutative square of categories of the following form
    has a \emph{unique} diagonal making the diagram commute :
    \begin{cd}
      {[0]} & X \\
      {[1]} & C
      \arrow["1"', from=1-1, to=2-1]
      \arrow[from=2-1, to=2-2]
      \arrow[from=1-2, to=2-2]
      \arrow[from=1-1, to=1-2]
      \arrow[dashed, from=2-1, to=1-2]
    \end{cd}
    We use $[n]$ to mean the linear order $0 \leq \cdots \leq n$
    viewed as a category.

    The outter square is called a \emph{lifting problem}
    and any such diagonal is called a \emph{solution}.
    \item the following is a $1$-categorical pullback diagram of categories : 
    \begin{cd}
      {X^{[1]}} & X \\
      {C^{[1]}} & C
      \arrow["\ev_1"', from=2-1, to=2-2]
      \arrow["{p^{[1]}}"', from=1-1, to=2-1]
      \arrow["\ev_1", from=1-1, to=1-2]
      \arrow["p", from=1-2, to=2-2]
      \arrow["\lrcorner"{anchor=center, pos=0.125}, draw=none, from=1-1, to=2-2]
    \end{cd}
  \end{enumerate}
  Define the \emph{category of discrete fibrations over $ C$} to be
  the full subcategory $\mathrm{Disc}/ C$ of $\CAT /  C$ consisting of
  discrete fibrations.
  With abuse of notation,
  we sometimes simply write $X$ to instead of $p : X \to C$.

  We then have the following : 
  \begin{enumerate}
    \item Let $\bullet/\SET$ be the category of (small) sets 
    under the singleton $\bullet$, 
    equivalently the category of pointed (small) sets.
    Then the forgetful functor $(\bullet/\SET)\op \to \SET\op$
    is a discrete fibration.
    \item The (1-categorical) pullback of 
    a discrete fibration is a discrete fibration.
    In particular, for any functor $F : C \to \SET\op$,
    we have 
    \begin{cd}
      {\Sigma F} & {(\bullet / \mathbf{Set})^{op}} \\
      C & {\mathbf{Set}^{op}}
      \arrow[from=1-2, to=2-2]
      \arrow["F"', from=2-1, to=2-2]
      \arrow[from=1-1, to=2-1]
      \arrow[from=1-1, to=1-2]
      \arrow["\lrcorner"{anchor=center, pos=0.125}, draw=none, from=1-1, to=2-2]
    \end{cd}
    where $\Sigma F \to C$ is a discrete fibration.
    The morphism  $\Sigma F \to C$ 
    can be explicitly defined :
    \begin{itemize}
      \item the objects of the category $\Sigma F$ are
      ``naive morphisms $c \to F$''
      and the morphisms between two naive morphisms $c \to F$ and $d \to F$
      are morphisms $f \in C(c,d)$ such that 
      ``the following triangle commutes'' :
      \begin{cd}
        c & d \\
        & F
        \arrow["f", from=1-1, to=1-2]
        \arrow[from=1-2, to=2-2]
        \arrow[from=1-1, to=2-2]
      \end{cd}
      \item the projection $\Sigma F \to C$ is given by taking
      the source of naive morphisms into $F$.
    \end{itemize}
    The above construction defines 
    a functor $\Sigma : (\SET\op)^C \to \mathrm{Disc}/C$.
    We will call $\Sigma F$ the \emph{total space of $F$}. 
    \item 
    We have a functor $\mathrm{Fib} : \mathrm{Disc}/ C \to (\SET\op)^C$
    defined by 
    \begin{itemize}
      \item On objects, take $(p : X \to  C)$ to the functor $C \to \SET\op$
      defined by
      \begin{itemize}
        \item on objects, $c$ is sent to $p\inv (c)$,
        which is a discrete category, i.e. a set.
        \item morphisms are defined by arrow-lifting definition of
        discrete fibrations.
        The uniqueness of the arrow-lifting gives functoriality.
      \end{itemize}
      \item For a morphism of discrete fibrations over $C$
      \begin{cd}
        X & Y \\
        & C
        \arrow["p"', from=1-1, to=2-2]
        \arrow["q", from=1-2, to=2-2]
        \arrow["f", from=1-1, to=1-2]
      \end{cd}
      fibers are mapped into fibers, thus giving
      a map of sets $f_c : \mathrm{Fib}(p)\, c \to \mathrm{Fib}(q)\, c$.
      Uniqueness of arrow-lifts for $(q : Y \to C)$
      implies $(f_c)_{c \in C}$ defines a natural transformation
      $\mathrm{Fib}(p) \to \mathrm{Fib}(q)$.
    \end{itemize}
    For $(p : X \to  C)$ in $\mathrm{Disc}/ C$,
    we call $\mathrm{Fib}(p)$ the \emph{fiber functor of $p$}.
    If the morphism $p$ is clear from the context,
    we will sometimes write $\mathrm{Fib} X$ instead.
    \item The functors $\mathrm{Fib}, \Sigma$ define
    an equivalence of categories
    \[
      \mathrm{Disc}/C \simeq (\SET\op)^C \iso \SET^{C\op}  
    \]
    In other words, we say that $(\bullet/\SET)\op \to \SET\op$ is the
    \emph{universal discrete fibration} and that 
    $\SET\op$ \emph{classifies} discrete fibrations.
  \end{enumerate}
  % We also have a functor $\int_C : \SET^{ C\op} \to \mathrm{Disc}/ C$ 
  % defined by : 
  
  % We will call $\int_C F$ the \emph{total space of $F$}.
  % It is also known as the \emph{Grothendieck construction}.

  % Then $\int_C$ and $\mathrm{Fib}$ give an equivalence of categories. 

\end{prop}
\begin{proof}
  \textit{(Equivalent definitions of discrete fibration)}
  \textit{$(2\implies 1)$} by taking objects.
  \textit{$(1\implies 2)$} 
  The assumption says that we have the pullback on the level of objects.
  To show the pullback for the level of arrows,
  note that this is equivalent to showing
  the lifting problems of the following form have \emph{unique} solutions :
  \begin{cd}
    {[1] \times\set{1}} & X \\
    {[1]\times [1]} & C
    \arrow[from=1-1, to=2-1]
    \arrow[from=1-2, to=2-2]
    \arrow[from=2-1, to=2-2]
    \arrow[from=1-1, to=1-2]
    \arrow[dashed, from=2-1, to=1-2]
  \end{cd}
  This can be thought of as a categorical version of
  the fact that in one can ``lift squares by an edge along a covering space''.
  Existence of the lift uses uniqueness of the arrow-lifts,
  and so does uniqueness.

  \textit{(1)} ok.

  \textit{(2)(Pullback of Discrete Fibrations)} 
  Via the unique arrow-lifting definition of discrete fibrations,
  the proof is direct. 
  Via the pullback definition of discrete fibrations,
  one can apply the pasting lemma for pullback squares.

  \textit{(2)($\Sigma$ defines a functor)}
  Let $h : F \to G$ be a natural transformation of functors
  $C \to \SET\op$.
  Then we the following commuting diagram of categories :
  \begin{cd}
    {\Sigma F + \Sigma G} & {\Sigma h} & {(\bullet/\mathbf{Set})^{op}} \\
    {C + C} & {C\times [1]} & {\mathbf{Set}^{op}}
    \arrow[from=1-3, to=2-3]
    \arrow["h"', from=2-2, to=2-3]
    \arrow[from=2-1, to=2-2]
    \arrow[from=1-2, to=2-2]
    \arrow[from=1-2, to=1-3]
    \arrow[from=1-1, to=2-1]
    \arrow[from=1-1, to=1-2]
    \arrow["\lrcorner"{anchor=center, pos=0.125}, draw=none, from=1-2, to=2-3]
    \arrow["\lrcorner"{anchor=center, pos=0.125}, draw=none, from=1-1, to=2-2]
  \end{cd}
  where $C + C$ denotes the coproduct of $C$ with itself and
  we have the left square is cartesian since
  the right and outter squares are cartesian (easy to check).
  One can check that the morphism $\Sigma F + \Sigma G \to \Sigma h$
  exhibits $\Sigma F, \Sigma G$ as full subcategories of $\Sigma h$.
  The unique-arrow lifting property of $\Sigma h$
  then defines as functor $\Sigma F \to \Sigma G$ over $C$.

  An alternate, more explicit construction is to use 
  the explicit definition of $\Sigma F \to C$.

  \textit{(3)} ok.

  \textit{(4)}
  Completely formal.
  The functors $\Sigma$ and $\mathrm{Fib}$ actually give
  an isomorphism of categories.
\end{proof}

\begin{dfn}[Presheaves over a Category]
  
  Let $ C$ be a (small) category.
  Then we will also call $\mathrm{Disc}/ C$ the
  \emph{category of presheaves over $ C$},
  denote it with $\PSH C$,
  and call its objects \emph{presheaves (over $ C$)}.
  For $c$ in $C$ and $X$ in $\PSH C$,
  we will also write $X(c)$ for the fiber of $X$ over $c$.
\end{dfn}

\begin{rmk}
  The standard definition of the category of presheaves on a category $C$
  is the category of functors $C\op \to \SET$.
  Our perhaps unconventional choice of definition of presheaves as
  discrete fibrations has the following advantages : 
  \begin{itemize}
    \item the proofs of all nice properties of $\PSH C$,
    i.e. topos-theoretic properties, are arguably more intuitive
    when phrased in terms of discrete fibrations.
    \item building intuition of presheaves as discrete fibrations
    makes the later generalisation to right fibrations of infinity categories
    obvious : 
    right fibrations are exactly discrete fibrations
    where sets are replaced with ``sets with equality as data''.
    Specifically,
    right fibrations over an infinity category $X$
    correspond to functors $X \to \infty\mathbf{Grpd}\op$
    to the infinity category of infinity groupoids.
  \end{itemize}
\end{rmk}

\begin{eg}
  
  Let $I$ be a discrete category, i.e. a set.
  Then presheaves over $I$ are the same data as
  a family of sets parameterised by $I$.
\end{eg}

\begin{eg}
  Let $C$ be a groupoid and $p : X \to C$ a morphism of categories.
  Then $p$ is a discrete fibration if and only if its opposite is.

  There exists an example of a category $C$ and a discrete fibration $X \to C$
  such that $X\op \to C\op$ is \emph{not} a discrete fibration.
\end{eg}

\section{Presheaf Categories closed under Slices - 
  Bundles and Relativisation}

\begin{prop}

  Let $C$ be a (small) category.
  Let $(p : S \to  C)$ in $\PSH C$.
  By composition with $p$, 
  we get a functor $\PSH S \to \PSH C / S$.
  This yields an equivalence of categories
  \[
    \PSH S \simeq \PSH C / S  
  \]
\end{prop}
\begin{proof}
  By our definition of presheaves over $ C$,
  this follows from the pasting lemma for pullback squares applied to
  a diagram of the form : 
  \begin{cd}
    {C^{[1]}} & {S^{[1]}} & {X^{[1]}} \\
    C & S & X \\
    {}
    \arrow[from=2-3, to=2-2]
    \arrow[from=2-2, to=2-1]
    \arrow[from=1-1, to=2-1]
    \arrow[from=1-2, to=1-1]
    \arrow[from=1-3, to=1-2]
    \arrow[from=1-2, to=2-2]
    \arrow[from=1-3, to=2-3]
    \arrow["\lrcorner"{anchor=center, pos=0.125, rotate=-90}, 
      draw=none, from=1-2, to=2-1]
  \end{cd}
  This is one of the reasons for this choice of definition of $\PSH C$.
\end{proof}

\section{Yoneda's Lemma - 
  Probeable objects are defined by how they can be probed}

\begin{prop}[``Test Objects are Clearly Probeable by Test Objects'']

  Let $ C$ be a (small) category.
  Then 
  \begin{enumerate}
    \item taking over-categories defines a functor
    $C \to \PSH C$.
    \item (Yoneda's lemma) Let $X \in \PSH C$ and $c \in  C$.
    Given a morphism $\al : C / c \to X$ of presheaves,
    we have a naive-morphism of $c$ to $x$ by taking
    the ``composition of $\al$ with $\id{c}$'',
    that is to say, the image of $\id{c}$ under $\al$.
    This defines the following bijection that is functorial in $c$ and $X$ : 
    \[
      \PSH C (C / c , X) \iso X (c)
    \]
    In other words, 
    ``morphisms from $c$ to $X$ as $C$-probeable spaces
    are the same as naive-morphisms from $c$ to $X$''.
    In particular,
    the functor $C \to \PSH C$ is fully faithful.
    This is called the \emph{Yoneda embedding}.
    Presheaves in its essential image are called \emph{representable}.

    Since morphisms $x : C / c \to X$ correspond to objects 
    in (the source category of) $X$,
    we also call these \emph{points of $X$}.
    For general morphisms $y : Y \to X$ from other presheaves,
    we call them \emph{generalised points of $X$}.
    \item (Density Theorem / "Function Extensionality for Presheaves") 
    Let $X$ in $\PSH C$.
    Consider the slice category $(C / \_ ) / X$
    consisting of Yoneda embeddings over $X$.
    This has an obvious structure as a presheaf over $C$
    and by the Yoneda lemma has a morphism $(C / \_)/ X \to X$ over $C$.
    In this way, we can convert $X$ into a diagram of presheaves over $C$.
    The result is that $X$ is the colimit of this diagram,
    written suggestively as $X = \COLIM_{x \in X} x$.
  \end{enumerate} 
\end{prop}
\begin{proof}
  \textit{(1)} The functor $C \to \PSH C$ acts on morphisms
  ``by composition''.

  \textit{(2)} 
  The hard part is realising that $C / c$ has $\id{c}$ as a final object.

  \textit{(3)} The point is that
  the data of a morphism out of $X \to Y$ is precisely 
  ``how to map the points and the morphisms'',
  and this is equivalent to a cocone structure of $Y$ over $(C / \_) / X$
  by Yoneda's lemma.
  
\end{proof}

\begin{eg}
  
  The density theorem applied to the case of $\SET = \PSH \bullet$
  recovers function extensionality.
\end{eg}

\section{Nerve and Realisation - Probing objects in other categories}

\begin{prop}

  Let $C$ be a (small) category and
  $F :  C \to D$ a functor.
  \begin{itemize}
    \item (Nerve)
    We can now probe objects in $D$ by objects in $C$ via $F$.
    This is formalised by the \emph{nerve functor} (with respect to $F$),
    which is defined as 
    \begin{align*}
      &N_F : D \to \PSH C \\
      & N_F \, d := F / d
    \end{align*}
    which can be defined by the pullback diagram : 
    \begin{cd}
      {F/d} & {D/d} \\
      C & D
      \arrow[from=1-1, to=2-1]
      \arrow[from=2-1, to=2-2]
      \arrow[from=1-2, to=2-2]
      \arrow[from=1-1, to=1-2]
      \arrow["\lrcorner"{anchor=center, pos=0.125}, draw=none, from=1-1, to=2-2]
    \end{cd}
    \item (Realisation and the Adjunction)
    There is an adjunction
    \[
      D ( \abs{\_}_F , \star) \iso \PSH C ( \_ , N_F \star)
    \]
    if and only if for every presheaf $X$ over $C$,
    the colimit \[
      \COLIM_{x \in X} F c  
    \]
    exists in $D$, where we see each $x \in X$ as a morphism $x : C / c \to X$
    as in the density theorem.
    In this case,
    $\abs{\_}_F$ is in fact the left Kan extension of $F$
    along the Yoneda embedding $C \to \PSH C$.

    The functor $\abs{\_}_F$ is called \emph{realisation} along $F$.

  \end{itemize}

  % \item (Free-Cocompletion)
  %   The category $\PSH C$ is cocomplete and 
  %   colimits are computed ``fiberwise''.
  %   Furthermore, we have an equivalence of categories 
  %   \[
  %     \mathrm{coComplete}(\PSH  C, -) \simeq \CAT( C, -)
  %   \]
  %   where the functor in the forwards direction is given by
  %   pre-composition with the Yoneda-embedding and 
  %   the functor in reverse is given by extending via colimits
  %   as in the density theorem.
  %   We say that $\PSH  C$ is the \emph{free cocompletion of $ C$}.
 
    % By the previous point, we have an extension of this functor
    % to a functor $\tilde{F} : \PSH C \to \DD$.
    % For $d$ in $\DD$, define the \emph{nerve of $d$ via $F$}
    % to be the object $N_F(d)$ in $\PSH C$ defined as the comma category $F / d$ 
    % equipped with the forgetful functor to $C$.
    % In other words, ``we probe $d$ by the test objects in $C$ using $F$''.
    % Composition by morphisms upgrade this into a functor
    % $N_F : \DD \to \PSH C$.
    % Then we have an adjunction $\tilde{F} \dashv N_F$.
\end{prop}
\begin{proof}
  
  \textit{(Adjunction)($\limplies$)}
  \begin{align*}
    D(\abs{X}_F , d) \iso D(\COLIM_{x \in X} F c , d)
    &\iso \LIM_{x \in X} D(F c , d) \\
    &\iso \LIM_{x \in X} \PSH C ( C / c , N_F d)
    \iso \PSH C (\COLIM_{x \in X} C / c , N_F d)
    \iso \PSH C (X , N_F d)
  \end{align*}

  $(\implies)$ Assuming we have such an adjunction,
  we have 
  \begin{align*}
    D(\abs{X}_F , d) \iso \PSH C (X , N_F d)
    \iso \PSH C (\COLIM_{x \in X} C / c , N_F d)
    \iso \LIM_{x \in X} \PSH C ( C / c , N_F d)
    \iso \LIM_{x \in X} D(F c , d)
  \end{align*}
  which shows $\abs{X}_F$ has the structure of
  a colimit desired.

  \textit{($\abs{\_}_F$ is the left Kan extension)}
  This comes down to the fact that
  the formula for the realisation is by colimits.

\end{proof}

\section{Pullback , Shriek Pushforward, Pushforward - 
  Reindexing, Dependent Sum, Dependent Products for Presheaves}

\begin{prop}

  For every functor $u : C \to D$,
  we have an adjoint triple of functors
  \begin{cd}
    {\mathbf{PSh} C} && {\mathbf{PSh} D}
    \arrow["{u^*}"{description}, from=1-3, to=1-1]
    \arrow["{u_*}"{description}, shift right=10, from=1-1, to=1-3]
    \arrow["{u_!}"{description}, shift left=10, from=1-1, to=1-3]
    \arrow["\bot"{description}, shift left=5, draw=none, from=1-1, to=1-3]
    \arrow["\bot"{description}, shift right=5, draw=none, from=1-1, to=1-3]
  \end{cd}  
  where the three functors are defined as follows : 
  \begin{itemize}
    \item (Shriek Pushforward / Dependent Sum)
    $u_!$ is given by realisation along the composition
    $C \to D \to \PSH D$.
    Recalling the formula for realisation, we have explicitly
    \[
      u_! X := \COLIM_{x \in X} D / u c
    \]
    where we view $x \in X$ as $x : C / c \to X$ as in the density theorem.

    We also write $\Sigma_u$ for $u_!$.

    In the functor perspective of presheaves,
    $u_! X$ is called the \emph{left Kan extension of $X$ along $u$}.

    \item (Pullback / Reindexing)
    $u^*$ is given by the nerve functor along
    $C \to D \to \PSH D$.
    Explicitly, this is given by the pullback of presheaves along $u$.

    In the perspective of presheaves as functors into $\SET\op$,
    $u^*$ corresponds to precomposition with $u$.
    \item (Pushforward / Dependent Product)
    $u_*$ is defined as the nerve functor along $u^* : D \to \PSH D \to \PSH C$.
    Explicitly,
    \[
      u_* X := \PSH C ( F / \_ , X)
    \]

    We also write $\Pi_u$ for $u_*$.

    In the functor perspective of presheaves,
    $u^* X$ is called the \emph{right Kan extension of $X$ along $u$}.
    
  \end{itemize}

  Hence, 
  all presheaf categories have colimits and limits.
  Furthermore, they are computed fiberwise in the sense that
  for any diagram $X : I \to \PSH C$ and $c \in C$,
  \begin{align*}
    (\COLIM_{i \in I} X)_c \iso \COLIM_{i \in I} (X_i)_c \\
    (\LIM_{i \in I} X)_c \iso \LIM_{i \in I} (X_i)_c
  \end{align*}
\end{prop}
\begin{proof}
  
  % \textit{(Dependent Sum)}
  % The goal is to show that for any $X \in \PSH C$
  % the colimit $\COLIM_{x \in X} D / u c$ exists and 
  % hence $u_!$ defines a functor.

  % Let $X \in \PSH C$.
  % We need to make a presheaf \emph{over $D$} from $X$.
  % We proceed by inspecting the diagram in question 
  % $\tilde{X} : (C/\_)/X \to \PSH D , x \mapsto D / u c$.
  % Naively, the presheaf we seek would have 
  % as fibers over each $d \in D$ in the image of $u$
  % the fiber of $X \to C \to D$ over $d$.
  % Though the fiber of $X \to C \to D$ over $d$
  % does give rise to a sub-diagram $\tilde{X}_d$ of $\tilde{X}$, 
  % this $\tilde{X}_d$ is hardly usually a discrete category,
  % which it must be since we want a presheaf.

  Let us first remark that the fact that limits of presheaves 
  is computed fiberwise is an easy consequence of Yoneda's lemma : 
  \[
    (\LIM_{i \in I} X_i)_c \iso 
    \PSH C (C / c , \LIM_{i \in I} X_i)
    \iso \LIM_{i \in I} \PSH C (C / c , X_i)
    \iso \LIM_{i \in I} (X_i)_c
  \]
  On the other hand,
  the fact that colimits of presheaves can be computed fiberwise
  is saying that representables are \emph{tiny}.

  A surprising fact I noticed is that
  the existence of the adjoint triple
  for all $u : C \to D$ where $C, D$ are (small) categories
  is actually equivalent to
  the cocompleteness of presheaf categories together with
  tininess of representables.

  First, let us assume we have the adjoint triple for all $u : C \to D$.
  Let $C$ be a (small) category.
  Then we can construct colimits and limits in $\PSH C$
  by noticing that given a category $I$,
  a diagram $X : I \to \PSH C$ is the same as 
  a functor $I \times C \to \SET\op$ by uncurrying,
  and hence is the same as the data of a presheaf $X \in \PSH (I \times C)$.
  It follows that colimits and limits of $I$-diagrams in $\PSH C$
  are giving by the dependent sum and dependent product
  along the projection functor $\si : I \times C \to C$ since
  pulling back along $\si$ corresponds exactly to 
  taking constant diagrams $\PSH C \to (\PSH C)^I$
  and we have the bijections,
  \begin{align*}
    \PSH C (\si_! X , \_ ) \iso \PSH(I \times C) (X , \si^* \_) \\
    \PSH (I \times C) (\si^* \_ , X) \iso \PSH C (\_ , \si_* X)
  \end{align*}
  Now for tininess of representables,
  the key observation is that for any object $c \in C$,
  pulling back along the corresponding functor $c : \bullet \to C$
  is the same as taking fiber over $c$.
  But we know that $c^* : \PSH C \to \PSH \bullet = \SET$
  has a right adjoint $c_*$,
  so $c^*$ must preserve both colimits.
  The fact that colimits and limits are computed fiberwise
  can be nicely phrased (though be it not very useful) as
  some sort of base change result
  \begin{align*}
    c^* \si_! &\iso \tau_! \ga^* \\
    c^* \si_* &\iso \tau_* \ga^*
  \end{align*}
  where
  \begin{cd}
    I & {I \times C} \\
    \bullet & C
    \arrow["\tau"', from=1-1, to=2-1]
    \arrow["\gamma", from=1-1, to=1-2]
    \arrow["c"', from=2-1, to=2-2]
    \arrow["\sigma", from=1-2, to=2-2]
    \arrow["\lrcorner"{anchor=center, pos=0.125}, draw=none, from=1-1, to=2-2]
  \end{cd}

  Now let's assume that $\PSH C$ is cocomplete
  and representables are tiny for all (small) $C$.
  Then by the nerve and realisation adjunction,
  we obtain the adjunction $u_! \dashv u^*$.
  We can also obtain the adjunction $u^* \dashv u_*$ by
  another application of the nerve-realisation adjunction
  provided we show that $u^*$ satisfies the formula : 
  \[
    u^* X \iso \COLIM_{x \in X} u^* D / d
  \]
  By tininess of representables in $\PSH C$,
  it suffices to check the fibers of $u^* X$ and 
  $\COLIM_{x \in X} u^* D / d$ match.
  But tininess of representables in $\PSH D$ give this : 
  \[
    (u^* X)_c = X_{u c}
    \iso \COLIM_{x \in X} (D / d)_{u c}
    = \COLIM_{x \in X} (u^* D / d)_c
    \iso (\COLIM_{x \in X} u^* D / d)_c
  \]

  Back to the proposition,
  by the above discussion 
  it suffices to show cocompleteness of presheaf categories and 
  tininess of representables.
  Well, given a diagram $X : I \to \PSH C$,
  we compromise the total space perspective and
  momentarily view presheaves as functors to define the presheaf
  \[
    \tilde{X}(c) := \COLIM_{i \in I} X_i(c)
  \]
  following the expected formula.
  The action of morphisms is given by UPs of the colimits.
  It is also easy to show that there is a cocone over $X$ with 
  tip $\tilde{X}$ making $\tilde{X}$ the colimit of $X$.
  By construction, this makes all representables tiny.
\end{proof}

\begin{eg}
  
  Let $I$ be a discrete category, i.e. a set,
  and $X \in \PSH I$, seen as a family of sets $(X_i)_{i \in I}$ over $I$.
  Let $p : I \to \bullet$ be the canonical morphism of sets to the singleton.
  Show that $\Sigma_p X$ is the disjoint union $\Sigma_{i \in I} X_i$
  and $\Pi_p X$ is the set of sections of $X$,
  i.e. $\Pi_{i \in I} X_i$.
\end{eg}

\section{Presheaf Category as the Free Cocompletion}

\begin{prop}
  
  Let $\mathrm{CoComplete}$ denote the category of
  cocomplete categories with cocontinuous functors as morphisms.
  Let $C$ be a category.
  Then we have an equivalence of categories : 
  \[
    \mathrm{CoComplete}(\PSH C , \_) \map{\sim}{}
    \mathbf{Cat}(C , \_)
  \]
\end{prop}
\begin{proof}
  We've seen that $\PSH C$ is cocomplete.
  Now let $i : \mathrm{CoComplete}(\PSH C , \_) \to \mathbf{Cat}(C , \_)$
  be obtained by precomposition with the Yoneda embedding $C \to \PSH C$.
  Full and faithfulness of $i$ follows from the density theorem.
  For essential surjectivity, 
  we know that any functor $F : C \to D$ where $D$ is cocomplete
  extends along the Yoneda embedding to $\abs{ \_ }_F : \PSH C \to D$ via
  the nerve-realisation adjunction.
  Since realisation is left adjoint to the nerve functor,
  it is cocontinuous.
\end{proof}

\section{Presheaf Categories are Cartesian Closed - 
  Mapping Objects and Currying}

\begin{prop}

  Let $C$ be a (small) category.
  Then $\PSH C$ is cartesian closed. 
  This means given two presheaves $X, Y$ over $C$,
  we have a presheaf $Y^X$ which ``internalises'' 
  the morphisms from $X$ to $Y$ in the sense that we have
  an adjunction reflecting \emph{currying} : 
  \[
    \PSH C (X \times (-) , Y) \iso \PSH C ( - , Y^X)  
  \]
\end{prop}
\begin{proof}
  Given a presheaf $X$ over $ C$,
  apply the nerve-realisation adjunction to the functor
  $X \times (-) : \PSH C \to \PSH C$.
\end{proof}

\section{Subobject Classifer - A universe of propositions}

\begin{prop}

  Let $C$ be a (small) category.
  For $X$ in $\PSH C$,
  define a \emph{subbundle of $p$} to be a subcategory of $X$
  which is also a presheaf over $C$.
  (Note that this implies subbundles are full subcategories.)
  Define the category $\SUB$ over $\PSH C$ via the data :
  \begin{itemize}
    \item the source category has as objects 
    subbundles of discrete bundles over $ C$.
    For morphisms between two subbundles $P, Q$ of 
    respectively $X, Y$,
    they are morphisms $f \in \PSH C(X,Y)$ such that
    we have an induced pullback square : 
    \begin{cd}
      P & Q \\
      X & Y
      \arrow[from=2-1, to=2-2, "f"{swap}]
      \arrow[from=1-1, to=2-1]
      \arrow[from=1-1, to=1-2]
      \arrow[from=1-2, to=2-2]
      \arrow["\lrcorner"{anchor=center, pos=0.125}, draw=none, from=1-1, to=2-2]
    \end{cd}
    In other words, that $P$ is the ``preimage'' of $Q$.
    \item there is a functor from the above source category to $\PSH C$
    by taking the ambient bundle of a subbundle.
  \end{itemize} 
  Then $\SUB$ defines a presheaf over $\PSH C$ and is in fact
  representable by its pullback to along $ C \to \PSH C$.
  We call its pullback to $C$ the 
  \emph{universe of propositions (of $\PSH C$)}
  and denote it with $\mathrm{Prop}$.
\end{prop}
\begin{proof}
  \textit{($\SUB \in \PSH C$)}
  We need to give a unique solution $\tilde{f}$ to
  the lifting problem 
  \begin{cd}
    {[0]} & {\mathbf{Sub}} \\
    {[1]} & {\mathbf{PSh}\mathcal{C}}
    \arrow["1"', from=1-1, to=2-1]
    \arrow["f"', from=2-1, to=2-2]
    \arrow["t", from=1-2, to=2-2]
    \arrow[from=1-1, to=1-2]
    \arrow["{\tilde{f}}"', dashed, from=2-1, to=1-2]
  \end{cd}
  What we have is two presheaves $X, Y$ over $ C$,
  a morphism $f : X \to Y$ over $ C$ and 
  a subbundle $S$ of $Y$.
  A lift $\tilde{f}$ is exactly a functor $X \to Y$ over $C$ 
  that is equal to $f$
  together with a subbundle of $X$ that is equal to 
  the preimage of $S$ under $f$.
  This tells us that the lift exists and is unique.

  \textit{(Representability of $\SUB$ by $\mathrm{Prop}$)}
  The goal is to give a bijection 
  \[
    \PSH C(X,\mathrm{Prop}) \iso \SUB(X)  
  \]
  that is functorial in $X$.
  The idea is that $\mathrm{Prop}$ is the ``universe of propositions''
  in $\PSH C$,
  so a morphism $X \to \mathrm{Prop}$ the same as a ``predicate on $X$'',
  which is exactly a ``subset'' of $X$, i.e. a subbundle.
  Thus, the forward map is defined as follows : 
  given a $P \in \PSH C(X,\mathrm{Prop})$,
  give the subbundle $\set{X\st P}$ of 
  ``points of $X$ where $P$ is true''.
  In other words, $\set{X\st P}$ is defined by the pullback diagram : 
  \begin{cd}
    {\{X\vert P\}} & \top \\
	  X & {\mathrm{Prop}}
	  \arrow["P"', from=2-1, to=2-2]
	  \arrow[from=1-1, to=2-1]
	  \arrow[from=1-2, to=2-2]
	  \arrow[from=1-1, to=1-2]
	  \arrow["\lrcorner"{anchor=center, pos=0.125}, draw=none, from=1-1, to=2-2]
  \end{cd}
  The \emph{universal subbundle} $\top \to \mathrm{Prop}$
  is defined by the full subcategory of $\mathrm{Prop}$
  consisting of inclusions of representables into themselves by identity,
  i.e. ``$\set{C / c \st \top}$'' across all $c$ in $ C$.
  
  We first show surjectivity.
  Let $S \to X$ be a subbundle of a presheaf over $ C$.
  We need to give a morphism $P \in \PSH C(X,\mathrm{Prop})$ such that
  we have the pullback square : 
  \begin{cd}
    S & \top \\
	  X & {\mathrm{Prop}}
	  \arrow["P"', from=2-1, to=2-2]
	  \arrow[from=1-1, to=2-1]
	  \arrow[from=1-2, to=2-2]
	  \arrow[from=1-1, to=1-2]
	  \arrow["\lrcorner"{anchor=center, pos=0.125}, draw=none, from=1-1, to=2-2]
  \end{cd}
  Intuition : $P$ is the predicate ``$\cdot \in S$''.
  Since $X$ is the colimit of its elements,
  to give a morphism $\text{``}\cdot \in S\text{''} : X \to \mathrm{Prop}$
  is the same as giving for each $x : C / c \to X$ 
  a morphism $C / c \to \mathrm{Prop}$
  that is functorial in the source $c$.
  The rest of the argument is contained in the following diagram
  together with the facts that limits are computed fiber-wise and that
  the correspondence we seek is trivial for the case of representables : 
  \begin{cd}
    {x^{-1}S} & S \\
    {C / c} & X & \top \\
    && {\mathrm{Prop}}
    \arrow[from=1-1, to=2-1]
    \arrow["x"', from=2-1, to=2-2]
    \arrow[from=1-1, to=1-2]
    \arrow["{\cdot \in S}"', dashed, from=2-2, to=3-3]
    \arrow["{\cdot \in x^{-1}S}"{swap}, bend right = 30, from=2-1, to=3-3]
    \arrow[from=1-2, to=2-2]
    \arrow["\lrcorner"{anchor=center, pos=0.125}, draw=none, from=1-1, to=2-2]
    \arrow[from=2-3, to=3-3]
    \arrow[dashed, from=1-2, to=2-3]
    \arrow[bend right = 30, from=1-1, to=2-3]
  \end{cd}

  Now to show injectivity, let $P, Q : X \to \mathrm{Prop}$
  such that $\set{X \st P} = \set{X \st Q}$.
  By ``function extensionality for presheaves'',
  it suffices to show that for all points $x$ of $X$ we have $P x = Q x$.
  The rest again follows from the fact that
  what we want to show is a tautology for the case of representables.

\end{proof}

\begin{eg}
  
  Already mentioned for motivation in the above proof, but
  for the case of $\SET = \PSH \bullet$,
  the set $\set{\nothing , \bullet}$ works as a subobject classifier.
  This is the categorical formalisation of the fact that
  in classical mathematics,
  ``all propositions are either true or false''.
\end{eg}

\section{Presheaf toposes - A place to do mathematics with strict equality}

\begin{rmk}[Presheaf Categories as a ``Place to Do Maths'']
  
  What the properties in the previous sections show is that
  presheaf categories have the categorical properties
  needed to interpret all we need to do ``maths'', 
  namely : 
  \begin{itemize}
    \item finite products to talk about contexts with multiple variables
    \item finite equalizers to talk about when ``elements'' are equal
    \item exponential objects to talk about morphisms
    \item a universe of propositions, to talk about ``predicates''
  \end{itemize}
  Abstracting these properties
  gives the definition of an \emph{elementary topos}.
  The canonical example is of course $\SET = \PSH\bullet$
  the category of presheaves over ``the abstract point''.
  
  % But this is not what we want,
  % since it is precisely strict equality of objects and morphisms
  % that we are struggling with!
  % The great idea that people had (quite recently!) is to
  % represent the ``data'' of an equality between two elements $s, t$
  % as a \emph{path} from $s$ to $t$.
  % And thus we enter the world of \emph{homotopy}.
\end{rmk}

% \begin{ex}

%   Define $\MOD \to \CRING$ where
%   \begin{itemize}
%     \item the category $\MOD$ has as objects $(A,M)$ where 
%     $A$ is in $\CRING$ and $M$ is in $A\MOD$,
%     and as morphisms between $(A,M) \to (B,N)$
%     pairs $(f,\phi)$ where
%     $f \in \CRING(A,B)$ and $\phi \in A\MOD(M,N)$
%     where $N$ is endowed with an $A$-module structure by
%     viewing $B$ as an $A$-algebra via $f$.
%     \item the functor $\MOD \to \CRING$ is takes $(A,M)$ to $A$.
%   \end{itemize}
%   Define \emph{the category of affine schemes} to be 
%   $\AFF := \CRING\op$ and 
%   $\QCOH := \MOD\op$ so that
%   $\QCOH \to \AFF$ is the opposite of the above functor.
%   For $A$ in $\CRING\op$, write $\SPEC A$ for the opposite of $A$.

%   Show that for any $A$ in $\CRING$, 
%   the fiber of $\MOD \to \CRING$ over $A$ is
%   equivalent to the category $A\MOD$,
%   and hence deduce that $\QCOH \to \AFF$ is \emph{not} a discrete fibration.
  
%   Show that for each $f \in \CRING(A,B)$ and $(A,M)$ in $\MOD$,
%   we have a lift $(A,M) \to (B, M \otimes_A B)$ of $f$.
%   Try to apply the (dual of the) fiber functor construction to
%   $\MOD \to \CRING$ to get a functor $\CRING \to \CAT$ to the category of 
%   (small) categories and realise that
%   functoriality is only true \emph{up to isomorphism} since
%   $(- \otimes_A B) \otimes_B C \neq (-) \otimes_A C$

%   This is an example of a ``bundle of categories''.
%   For (1-)categories, they go by the name of \emph{Grothendieck fibrations},
%   or \emph{fibered categories}.
%   A systematic way of dealing with
%   ``commuting up to isomorphisms''
%   is a way of motivating infinity-category theory.

% \end{ex}

\end{document}
